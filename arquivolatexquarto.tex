\documentclass[12pt, a4paper, oneside]{book}

\usepackage[T1]{fontenc} % pacote define aleitura do texto diretamente pelo teclado
\usepackage[utf8]{inputenc}
\usepackage[brazilian]{babel}
\usepackage[top = 2cm, bottom = 2cm, left = 2.5cm, right = 2.5cm]{geometry} % margem
\linespread{1.0} %espaço entre as linhas
\usepackage{amsmath,array,amssymb}
\usepackage[dvipsnames, svgnames,xllnames ]{xcolor}
\usepackage{setspace}
\usepackage{graphicx}
\usepackage{color}
\usepackage{afterpage}
\usepackage{indentfirst}

\begin{document}

tabela 1
\begin{tabular}{|l|c|r|} \hline
Célula 1 & Célula 2 & Célula 3 \\ \hline
Célula 4 & Célula 5 & Célula 6 \\ \hline
Célula 7 & Célula 8 & Célula 9 \\ \hline
\end{tabular}

\vspace{2cm}
tabela 2
\begin{tabular}{|c|c|c|c|}\hline
\multicolumn{4}{|c|}{Meses do ano}\\ \hline 
Janeiro & fevereiro & Março & Abril\\ \hline
Maio & Junho & Julho & Agosto\\ \hline
Setembro & Outubro & Novembro & Dezembro\\ \hline
\end{tabular}

\vspace{2cm}
tabela 3\\
\vspace{2cm}
\begin{tabular}{|c|cc|} \hline 
Números & \multicolumn{2}{|c|}{Meses do Ano e Abreviação}\\ \hline
1 & Janeiro & Jan\\ \cline{2-3} % cline cria a linha horizontal que liga a coluna 2 a 3
2 & Fevereiro & Fev\\ \cline{2-3}
3 & Março & Mar\\ \cline{2-3}
4 & Abril & Abr\\ \cline{2-3}
5 & Maio & Mai\\ \cline{2-3}
6 & Junho & Jun\\ \hline
\end{tabular}

ambiente table e tabular//

\begin{table}[h]
\centering  %centailizar na folha
\caption{maiores paises do mundo em extensão} % comando para título da tabela
\vspace{0.5cm}
\begin{tabular}{c|cc}
Posição & Paises & Extensão Territorial ($km^{2}$)\\
\hline
1 & Rússia &          17.098.246\\
2 & Canadá &           9.984.670\\
3 & China &            9.596.961\\
4 & Estados Unidos &   9.371.174\\
5 & Brasil &           8.515.767\\
\end{tabular}
\end{table}

\vspace{2cm}
Como colocar figura no projeto - não esquecer de colocar o pacote graphicx no preâmbulo
\vspace{2cm}


\begin{figure}[h]
 \centering
 \caption{imagem de uma lâmpada}
 \includegraphics[scale=1]{C:/Users/Eu/Pictures/lampada.png} % scale define tamanho
 \label{imagemlampada}
  \end{figure}
  
\newpage
  
\vspace{2cm}
  MINIPAGE - página de rosto
  
\vspace{2cm}

\begin{minipage}{10cm}
"Imagine quando seu professor pede para você escrever um estudo de caso. Não tem idéia de como formatar um estudo de caso. O Fastformat resolve isso para você falando o que deve ser escrito em cada uma das seções do estudo. Isso tudo já se encontra disponível no Fastformat quando você utiliza a funcionalidade de “Modelos“.
\end{minipage} \newpage



\mbox{} % permite usar vspace para descer o texto
\vspace{15cm}

\begin{flushright}
\begin{minipage}{10cm}
\hrulefill

"Imagine quando seu professor pede para você escrever um estudo de caso. Não tem idéia de como formatar um estudo de caso. O Fastformat resolve isso para você falando o que deve ser escrito em cada uma das seções do estudo. Isso tudo já se encontra disponível no Fastformat quando você utiliza a funcionalidade de “Modelos“.

\hrulefill

\textbf{Orientador: Prof. Dr: Fulano} 
\end{minipage}
\end{flushright} \newpage

fazer capa

\begin{titlepage}
\addtolength{\topmargin}{1.5cm} % distancia do topo da página para primeiro elemento da capa

\setlength{\baselineskip}{1.4\baselineskip} 

\begin{center}
\large{UNIVERSIDADE FEDERAL DE GOIÁS}
%tem que deixar um espaço

\large{FACULDADE DE ADMINISTRAÇÃO}
\end{center}


\vspace{2cm}

\begin{center}
\Large\textbf{Organizações Empreendedoras são Saudáveis?}
\end{center}

\vspace{1,5cm}
\begin{center}
\large(Autor do trabalho
\end{center}

\vspace{1,5cm}

\begin{flushright}
\begin{minipage}{12cm}
\hrulefill

Trabalho final de graduação do curso de Administração da Universidade Federal de Goiás apresentado como pré requisito para a obtenção do grau de bacharelado em Administração.

\hrulefill\\

\textbf{Orientador: Prof. Dr. Fulano}
\end{minipage}
\end{flushright}

\setlength{\baselineskip}{0.7\baselineskip} 
\vfill % coloca um espaço ajusrtado para não usar vspace

\begin{center}
Goiânia

Junho de 2023.
\end{center}
\end{titlepage}

\vspace{1,5cm}

RESUMOS E AGRADECIMENTOS

\chapter*{Resumo}

\noindent
Aqui vai o texto do resumo\\
\noindent
\textbf{Palavras-Chaves:}

\chapter*{Agradecimentos}
\noindent
Aqui vai os agradecimentos as pessoasmais proximas ao trabalho

\end{document}

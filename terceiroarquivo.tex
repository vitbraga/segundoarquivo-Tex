\documentclass[12pt, a4paper, oneside]{article}

\usepackage[T1]{fontenc} % pacote define aleitura do texto diretamente pelo teclado
\usepackage[utf8]{inputenc}
\usepackage[brazilian]{babel}
\usepackage[top = 2cm, bottom = 2cm, left = 2.5cm, right = 2.5cm]{geometry} % margem
\linespread{2.0} %espaço entre as linhas
\usepackage{amsmath,array, amssymb}

\begin{document}

\title{Capítulo 4 Ambiente matemático}
\author{Vitória Augusta Braga de Souza \footnote{doutora em Engenharia do conhecimento}}
\date{23 de Janeiro de 2023}
\maketitle % criação outomatica de título autor e data

segundo a equação $ x=2 $, x está valendo 2.
abaixo utilizando o ambiente equation: 
\begin{equation}
x=2.
\end{equation}
 a equação fica centraalizada e já numerada
 
 \begin{equation}
 \frac{a}{b}
 \end{equation}
 
 \begin{equation}
 3^{2}
 \end{equation}
 
 \begin{equation}
 \sqrt[3]{27}
 \end{equation}
 
 \begin{equation}
 \log_{2} 4
 \end{equation}
 
 equação quimica escreve assim:
 \begin{equation}
 6 CO_{2} = 6 H_{2}O + calor \rightarrow 6 O_{2} + C_{6}H_{12}O_6
 \end{equation}
 
 AS EXPRESSÕES TRIGONOMÉTRICAS SÃO ESCRITAS ASSIM:
 $\sin 60°$
 ou assim:
\begin{equation}
\ cos 60°.
\end{equation}

As funçõe são escritas assim:
\begin{equation}
f(x)=\sqrt[3]{x} + 2x + x^{2} 
\end{equation}

\begin{equation}
f(x)= 2x^{2} + x + 4
\end{equation}

\begin{equation}
\left( \frac{a}{b}\right) 
\end{equation}

%limites - limite de x tendendo a 1 de x ao cubo menos 3

\begin{equation}
\lim_{x\rightarrow 1}(x^{3}- 3)
\end{equation}

\begin{equation}
\lim_{x\rightarrow 2} \sqrt{(x^{4}-8)}
\end{equation}
 
 \begin{equation}
 \lim_{x\rightarrow -3} \frac{x^{2}-9}{x+3}
 \end{equation}
 
 \begin{equation}
 \lim_{x\rightarrow\infty} \frac{1}{2x}
 \end{equation}

\begin{equation}
\int(e^{-x} + 2^{x})dx
\end{equation}

\begin{equation}
\int^{b}_{a} f(x)dx = F(b)- F (a)
\end{equation}
%se colocar o _{a} antes não altera //

 lei da gravitação universal Newton - uso do vetor
\begin{equation}
\vec{F} = -G\frac{m_1m_2}{r^2}\hat{r}
\end{equation}

\begin{equation}
6,6 \times 10^{-11} \frac{m^{3}}{Kg^{-1} 8^{-2}•}
\end{equation}

\begin{equation}
f(t)= \frac{1}{2} + \frac{\cos\frac{\pi}{3}}{2\pi}\sum_{-\infty} ^{\infty} \frac{1}{n} e^{Bn2\pi t}
\end{equation}
% tem que colar espaço ente o comando e a letra em sequencia
\vspace{1,0cm}

\begin{equation}
\frac{a}{b+\frac{b+1}{c+\frac{c+1}{d+\frac{d+1}{e}}}}
\end{equation}

%\lbrace é o comando da chaves ou usa as chaves do teclado

\vspace{1,0cm}

\begin{equation}
\frac{d}{dt}\left( mr^{2} \frac{d\theta}{dt}\right) = 0
\end{equation}
\vspace{1,0cm}
%matriz 

\begin{equation}
\left(
\begin{array}{lr}
 a & b \\
 c & d \\
 
\end{array}
\right) 
\end{equation}

\vspace{1,0cm}

\begin{equation}
\left(
\begin{array}{lcr}
 a & b & c \\
 d & e & f \\
 g & h & i \\
\end{array}
\right) 
\end{equation}

\vspace{1,0cm}
%ambiente pmatriz. Para funcionar é necessário acrescentar no preambolo os pacotes {amsmath,array, amssymb}
\begin{equation}
\begin{pmatrix}
 x & y & z\\
 w & h & x\\
\end{pmatrix}
\end{equation}

\vspace{1,0cm}

\begin{equation}
\begin{pmatrix}
 x & y & z & i\\
 w & h & r & h\\
 r & t & j & g\\
\end{pmatrix}
\end{equation}
  
\vspace{1,0cm}
% sistema linear array

\begin{equation}
\left\lbrace % só uma chave 
\begin{array}{cc}% cc centralizado
      3x+2y =6\\
      2x+3y = 5
\end{array}
\right. % não aparece a outra chave
\end{equation} 

\vspace{1,0cm}

\begin{equation}
\left\lbrace
\begin{array}{ccc}
  x + y + z = 6\\
  x + 2y + 2z =9\\
  2x + y + 3z = 11
\end{array}
\right.
\end{equation}

% sistema linear eqnarray -equações independentes sem chaves

\begin{eqnarray}
      3x+2y =6\\
      2x+3y = 5
\end{eqnarray}

\vspace{1,0cm}

\begin{eqnarray}
x + y + z = 6\\
  x + 2y + 2z =9\\
  2x + y + 3z = 11
\end{eqnarray}

\vspace{1,0cm}

% sistema linear condicional

\begin{equation}
f(x)=
\left\lbrace
\begin{array}{cc}
x - 1, & x = 2 \\
2x +3, & x \neq 2
 \end{array}
\right.
\end{equation}

\vspace{1,0cm}
\begin{center}
detreminantes
\end{center}

\begin{equation}
det(A)= 
\left\vert
\begin{array}{lcr}
a_{11} & b_{12} & c_{13}\\
d_{21} & e_{22} & f_{23}\\
g_{31} & h_{32} & i_{33}\\
\end{array}
\right\vert
\end{equation}

\vspace{1,0cm}

\begin{eqnarray} %vspace não funciona no ambiente matemático
\overline{(A \cdot B)}\\
\overline{( A + B + C) \cdot \left(\frac{A}{B})\right)}\\
\vec{F}= \overline{A \cdot B \cdot C} + \overline{(A-B-C)}
\end{eqnarray}

\vspace{1,0cm}

para colocar espaço ente os elementos da expressão usar \ e "," ou ";" ou ";"
\vspace{1,0cm}
\begin{equation}
y(0,1\,m,30\,s) = (0,05\,m)\,sen(1 - 1500)\Rightarrow (0,05\, m)\,sen\,(-1499)
\end{equation}

\vspace{1,0cm}
Código fonte stackrel - colocar um elemento sobre o outro
\begin{equation} 
A + B \;\stackrel{2\,mim}{\longrightarrow}\;C + D
\end{equation}

\end{document}
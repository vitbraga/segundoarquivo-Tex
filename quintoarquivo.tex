\documentclass[12pt, a4paper, oneside]{book}

\usepackage[T1]{fontenc} % pacote define aleitura do texto diretamente pelo teclado
\usepackage[utf8]{inputenc}
\usepackage[brazilian]{babel}
\usepackage[top = 2cm, bottom = 2cm, left = 2.5cm, right = 2.5cm]{geometry} % margem
\linespread{1.0} %espaço entre as linhas
\usepackage{amsmath,array,amssymb}
\usepackage[dvipsnames, svgnames,xllnames ]{xcolor}
\usepackage{setspace}
\usepackage{graphicx}
\usepackage{color}
\usepackage{afterpage}
\usepackage{indentfirst}
\usepackage{subfigure}
\begin{document}

REFERENCIANDO EQUAÇÔES
\vspace{1,5cm}

\begin{equation}\label{Eqtrigonometria}
\sin^{2}\alpha +\cos^\alpha = 1
\end{equation}
 A equação \ref{Eqtrigonometria} é a primeira relação fundamental da trigonometria.
 
 \newpage
 
\vspace{1,5cm}
%INSERIR SUBFIGURAS  precisa do pacote subfigure no preambulo

\begin{figure}
\centering

\subfigure{(a)Planeta Terra \label{Figterra}}\includegraphics[scale=0.5]{C:/Users/Eu/Pictures/terra.png}
\subfigure{(b)Planeta Marte \label{Figmarte}}\includegraphics[scale=0.5]{C:/Users/Eu/Pictures/marte.png}
\subfigure{(c)Planeta Saturno \label{Figsaturno}}\includegraphics[scale=0.5]{C:/Users/Eu/Pictures/saturno.png}
\caption{Alguns planetas do Sistema Solar}\label{figSistemaSolar}

\end{figure}

\vspace{0.2cm}

Nas imagens acima, \ref{Figterra} representa o planeta Terra, \ref{Figmarte} o planeta vermelho Marte e \ref{Figsaturno} oplaneta Saturno. A imagem \ref{figSistemaSolar}, portanto comtém 3 planetas do Sistema Solar.


%SÚMÁRIO
\tableofcontents
\newpage

\chapter{Introdução}

\section{Assunto 1}

\section{Assunto 2}

\chapter{Desenvolvimento}

\section{Assunto 3}

\section{Assunto 4}

\chapter{Conclusão}

\section{Assunto 5}







\end{document}